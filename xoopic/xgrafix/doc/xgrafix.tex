\documentstyle[12pt]{article}
\pagestyle{plain}
\topmargin -0.60in
\oddsidemargin 0.0625in
\textheight 9.00in
\textwidth 6.50in
\renewcommand{\baselinestretch}{1.4}
\parskip 0.10in

\begin{document}

\begin{center}
\Large
{\bf XGrafix library} \\
Version 2.0
\\
\large
V. Vahedi, P. Mirrashidi, D. K. Wong, and J. P. Verboncoeur \\
{\em University of California, Berkeley}
\end{center}

\normalsize
XGrafix is a windowing environment for interactive display of
results from any temporally evolving simulation of a physical
system.  XGrafix is distributed in library format by the University
of California Office of Technology Licensing, Berkeley California.
The source codes to the libraries are also distributed separately
by the Office of Technology Licensing.  The README file with the
source code describes how to compile the XGrafix library using
the distributed makefile.  Before running make, the flags and
variables in the makefile must be defined. After compiling XGrafix,
the makefile builds libraries which can be left in the XGrafix
directory or installed in {\bf /usr/lib}.  Two header files are
included in the distribution, only one of which is to be included
in the physics modules.  The file xgrafix.h contains function prototypes
and variable definitions that may be needed.  The other header file, 
xgrafixint.h, is used only to compile XGrafix and is not needed by 
the physics modules.

Another directory called {\bf ctest} is also
distributed with the XGrafix library to demonstrate how to use
XGrafix with C compilers. We will
describe the usage of XGrafix using the example in the {\bf ctest}
directory.

Another directory called {\bf ftest} is also distributed to
demonstrate how to use XGrafix with Fortran compilers. 
The usage of XGrafix with Fortran compilers is very similar
to its usage with C compilers. First follow the
explanation of the example given in {\bf ctest} before
trying the example in {\bf ftest}. When compiling Fortran programs with
the XGrafix libraries you will need the file {\bf extra.c} provided in {\bf ftest}.
Also, Depending on your system additional Fortran compilation flags and Fortran
libraries may be needed. This is because the implementation of the Fortran 
procedure $getarg()$ is system-dependent. For example, Sun machines will require
the library libF77.a while HP machines will require the compilation flag +U77 and
the library libU77.a. For more information, do a ``man getarg'' on your system. 

The XGrafix function calls in Fortran are nearly identical to those in C, 
except the functions are all in lowercase. For each C Xgrafix function,
there is a corresponding Fortran function which takes virtually identical
parameters. There are a few exceptions to this rule:

\begin{enumerate}
\item Unlike the C Function $XGInit()$, the Fortran function $xginit()$ 
has only one parameter: the time.

\item A Fortran equivalent is provided for $XGStructureArray()$ but not for
$XGStructure()$ because $XGStructureArray()$
does the same thing as $XGStructure()$ but with parameters which
are more compatible with Fortran. Also, in the Fortran version of
$XGStructureArray()$, the  parameters FILLED, HOLLOW are replaced
by the integers 0,1 respectively.

\item Fortran equivalents are provided for $XGCurve()$, $XGVector()$, $XGSurf()$ and \\ 
$XGIRSurf()$ but not for $XGCurveVector()$, $XGVectorVector()$,
$XGSurfVector()$ and $XGIRSurfVector()$. This is because the latter set of functions 
are used only when plotting 
data structures specific to C and not relevant to Fortran.

\item The Fortran functions $xgvector()$, $xgsurf()$, $xgirsurf()$ 
and $xgcont()$ are nearly identical to their C counterparts except that they 
take on two additional parameters: The dimensions of the 2D array they are 
passing. These should be inserted right before the "color" parameter.  

\item Instead of using the C function $SetupNewVar()$ to set up variables which can be 
changed dynamically by XGrafix, use the functions $setupnewint(integer, string)$, \\
$setupnewreal(real, string)$, $setupnewdouble(double precision, string)$, and \\
$setupnewchar(character, string)$. In each case,
``string'' is the label XGrafix will associate with the dynamic variable. 
Like $SetupNewVar()$, these functions must be called before $XGStart()$.
\end{enumerate}
For more information on the Fortran Xgrafix functions, read the comments in the file \\
{\bf xgfinterface.f} in the {\bf src} directory.
For usage of these functions, see the file {\bf test.f} in the {\bf ftest} directory.

XGrafix allows the application to initialize plotting windows for
displaying data.  All the arrays to be plotted must remain defined for
the duration of the simulation run.  XGrafix is initialized by calling
the {\em XGInit} () function before any plots can be set up.  In the
test example, the call to {\em XGinit} () appears in the function {\em
InitWindows} () which precedes the main event loop.  Definitions of
{\em XGInit} () and other XGrafix functions will appear at the end of
this document.  After XGrafix is initialized, subsequent calls to {\em
XGSet2D} (), {\em XGSet2DC} (), {\em XGSetVec} (), and {\em XGSet3D}
() initialize the appropriate plotting windows.  In the sample, these
functions are called in the {\em InitWindows} () function.  After a 2D
plotting window is initialized with a call to {\em XGSet2D} () the
application can request multiple graphs in the plotting window by
successive calls to {\em XGCurve} (), {\em XGCurveVector} () and/or {\em
XGScat2D} ().  This version of
XGrafix supports only one graph in any 3D plotting window set up by
{\em XGSet3D} ().

\newpage
Simulations of physical systems which evolve in time, typically have
a time loop in which various physics module are called and the
time step and various data arrays are updated.  XGrafix automatically handles
this loop and updates the physics and graphics.  An {\em XGMainLoop} () 
procedure needs to be supplied, which would update variables as needed for 
each iteration.  A call to {\em XGStart} () after all the initializations
are done will enter this loop and take over.
The functions listed below are XGrafix routines to
initialize the environments and set up windows for diagnostics.

\def\xlen{4in}
\begin{minipage}{\xlen}
\begin{flushright}
\rule{\xlen}{.5pt}
\end{flushright}
\end{minipage}

\begin{flushleft}
{\em XGInit} ($argc$, $argv$, $t$) \\
int $argc$; /* Number of arguments passed to the $main$ () */\\
char *$argv$ [ ]; /* Array of char pointers to list of input parameters on command line */ \\
double *$t$;  /* a pointer to the time variable */
\end{flushleft}
This function must be called to initialize XGrafix before any other XGrafix routine is
called.  The parameters $argv$ and $argc$ are the standard parameters passed to
$main$ () at the command line.  The pointer to the time variable $t$ is 
used by XGrafix to display the simulation time.  XGrafix uses argv[0] to 
obtain the name of the executable program.  XGrafix also allows for several command line options.  The $-h$ option will output a list of other supported options:

\begin{tabbing}
\verb+    +\=$-d$ $dumpfile[.dmp]$: \verb+ +\=Specify dump file name. Default is output.dmp \\
\>$-dp$ $n$:\>Dump every $n$ steps. Default is 0 (no periodic dumps) \\
\>$-h$:\>This help message \\
\>$-i$ $inputfile[.inp]$:\>Specify input file name. Default is no input file \\
\>$-nox$:\>Run without X \\
\>$-p$ $psfile[.ps]$:\>Specify ps file name. Default is output.ps \\
\>$-s$ $n$:\>Run for $n$ steps. Default is 0 (run forever) \\
\>$-u$ $n$:\>Run $n$ XGMainLoops every graphics update.  Default is 1. \\
\end{tabbing}

\newpage
After the call to XGInit, several variables will be available to the physics 
module.  These are:

\begin{tabbing}
\verb+    +\=int\verb+    +\=$theRunWithXFlag$\verb+    +\=FALSE if $-nox$ is given, TRUE otherwise \\
\>int\>$theNumberOfSteps$\>The number of steps simulation will run, 0 if forever \\
\>int\>$theCurrentStep$\>The current step in the simulation \\
\>int\>$theDumpPeriod$\>The number of steps between automatic dumps \\
\>int\>$WasDumpFileGiven$\>TRUE if $-d$ option is given, FALSE otherwise \\
\>int\>$WasInputFileGiven$\>TRUE if $-i$ option is given, FALSE otherwise \\
\>char *\>$theCodeName$\>The name of the physics executable, same as argv[0] \\
\>char *\>$theInputFile$\>The name of the input file \\
\>char *\>$theDumpFile$\>The name of the dump file \\
\>char *\>$theEPSFile$\>The name of the encapsulated postscript file
\end{tabbing}

\def\xlen{4in}
\begin{minipage}{\xlen}
\begin{flushright}
\rule{\xlen}{.5pt}
\end{flushright}
\end{minipage}

\begin{flushleft}
$XGSet2D$($PlotType$, $X\_Label$, $Y\_Label$, $State$, $ulx$, $uly$, $X\_Scale$, $Y\_Scale$, \\
\hspace{.8in} $X\_Auto\_Rescale$, $Y\_Auto\_Rescale$, $X\_Min$, $X\_Max$, $Y\_Min$, $Y\_Max$) \\
char  *$PlotType$;         /* ``linlin'', ``linlog'', ``loglog'', or ``loglin'' */ \\
char  *$X\_Label$;         /* x label for the frame                            */ \\
char  *$Y\_Label$;         /* y label for the frame                            */ \\
char  *$State$;            /* ``open'' or ``closed''                           */ \\
int    $ulx$, $uly$;       /* requested position of frame's upper left corner   */ \\
SCALAR  $X\_Scale$;         /* scaling factor for the x array                   */ \\
SCALAR  $Y\_Scale$;         /* scaling factor for the y array                   */ \\
int    $X\_Auto\_Rescale$; /* if True $X\_Min$ and $X\_Max$ are ignored       */ \\
int    $Y\_Auto\_Rescale$; /* if True $Y\_Min$ and $Y\_Max$ are ignored       */ \\
SCALAR  $X\_Min$, $X\_Max$; /* x bounds for the plot if $X\_Auto\_Rescale$ is False  */ \\
SCALAR  $Y\_Min$, $Y\_Max$; /* y bounds for the plot if $Y\_Auto\_Rescale$ is False  */ 
\end{flushleft}

{\em XGSet2D} () initializes a plotting frame for a 2D plot.  The plot type is passed 
as a character string and can be one of the four strings defined in the comments.  If
$PlotType$ is defined as ``linlog,'' the $x$ axis will have a linear scale, and the $y$
axis will have a logarithmic scale.  Note that when the $x$ and/or $y$ axes are set to have
a logarithmic scale, XGrafix WILL take the logarithm of the data along that axis.
The name of the X-Windows frame, which appears as a window on the display will be
the same $Y\_Label$.  The $State$ parameter (``open'' or ``closed'') defines the
initial state of the window; if ``closed,''
the window will initially be hidden.  If the flag $X\_Auto\_Rescale$ is
set to zero (False), XGrafix uses $X\_Min$ and $X\_Max$ as the plot limits on the $x$
axis in the plotting window.  If $X\_Auto\_Rescale$ is not False, set to 1, XGrafix will
scan the whole $x$ data array to find the maximum and the minimum of each plot.  The 
same is true for the parameters pertaining to $y$ axis.  Note that, {\em XGSet2D} ()
only sets up a 2D frame, and does not have any information about the data arrays which
are to be plotted in this frame.  A call to {\em XGSet2D} () must be followed by at
least one call to $XGCurve$ (), $XGCurveVector$ () or $XGScat2D$ () .  Multiple calls to $XGCurve$ (), $XGCurveVector$ ()  
and/or $XGScat2D$ () result in multiple graphs in that plotting frame. The integers
$ulx$ and $uly$ define the initial position of frame's upper left corner.  The default
size of a 2D frame is 300 by 400 pixels which can be resized at run time.

\begin{flushleft}
$XGCurve$($x\_array$, $y\_array$, $npoints$, $color$) \\
SCALAR   *$x\_array$;    /* x array to be plotted                         */ \\
SCALAR   *$y\_array$;    /* y array to be plotted                         */ \\
int     *$npoints$;    /* number of points in the x (and y) direction    */ \\
int      $color$;      /* plot color chosen from 0-9                   */
\end{flushleft}

\begin{flushleft}
$XGCurveVector$($x\_array$, $y\_array$, $npoints$, $color$, $x\_size$,
$x\_offset$, $y\_size$, $y\_offset$) \\
SCALAR   *$x\_array$;    /* x array to be plotted                         */ \\
SCALAR   *$y\_array$;    /* y array to be plotted                         */ \\
int     *$npoints$;    /* number of points in the x (and y) direction    */ \\
int      $color$;      /* plot color chosen from 0-9 			*/ \\
int      $x\_size$;      /* number of SCALARs in the stucture which
contains the x data	*/ \\
int      $x\_offset$;      /* number of SCALARs preceding the x field
in the structure containing the x data		*/ \\
int      $y\_size$;      /* number of SCALARs in the structure which
	contains the y data*/ \\
int      $y\_offset$;      /* number of SCALARs preceding the y field
 in the structure containing the y data*/
\end{flushleft}

\begin{flushleft}
$XGScat2D$($x\_array$, $y\_array$, $npoints$, $color$) \\
SCALAR   *$x\_array$;    /* x array to be plotted                          */ \\
SCALAR   *$y\_array$;    /* y array to be plotted                          */ \\
int     *$npoints$;    /* number of points in the x (and y) direction    */ \\
int      $color$;      /* plot's color chosen from 0-9                   */
\end{flushleft}

These functions add a curve to an existing plotting frame.  Each call
to $XGCurve$(), $XGCurveVector$ ()  or $XGScat2D$() associates the curve or scatter plot
with the plot frame corresponding to the preceding call to $XGSet2D$().
Note that each frame can contain one or more curves or scatter plots, in
any combination.

{\bf Note} that the $x\_array$, $y\_array$, and $npoints$ are shared memory.
Therefore, once the pointers have been passed, these memory must not be freed
or moved until program termination. 

{\bf Also note} that ``SCALAR'' is set to either ``float'' or
``double'' during the compilation of XGrafix.  It is defined in the
files xgrafix.h and xgdatamacros.h.

\def\xlen{4in}
\begin{minipage}{\xlen}
\begin{flushright}
\rule{\xlen}{.5pt}
\end{flushright}
\end{minipage}

\begin{flushleft}
$XGSetVec$($PlotType$, $X\_Label$, $Y\_Label$, $State$, $ulx$, $uly$, $X\_Scale$, $Y\_Scale$, \\
\hspace{.8in} $X\_Auto\_Rescale$, $Y\_Auto\_Rescale$, $X\_Min$, $X\_Max$, $Y\_Min$, $Y\_Max$) \\
char  *$PlotType$;         /* only ``vecvec'' is permitted*/ \\
char  *$X\_Label$;         /* x label for the frame                            */ \\
char  *$Y\_Label$;         /* y label for the frame
*/ \\
char  *$Z\_Label$;           /* z label for the frame (name of vectors)                            */ \\
char  *$State$;            /* ``open'' or ``closed''                           */ \\
int    $ulx$, $uly$;       /* requested position of frame's upper left corner   */ \\
SCALAR  $X\_Scale$;         /* scaling factor for the x array                   */ \\
SCALAR  $Y\_Scale$;         /* scaling factor for the y array                   */ \\
int    $X\_Auto\_Rescale$; /* if True $X\_Min$ and $X\_Max$ are ignored       */ \\
int    $Y\_Auto\_Rescale$; /* if True $Y\_Min$ and $Y\_Max$ are ignored       */ \\
SCALAR  $X\_Min$, $X\_Max$; /* x bounds for the plot if $X\_Auto\_Rescale$ is False  */ \\
SCALAR  $Y\_Min$, $Y\_Max$; /* y bounds for the plot if $Y\_Auto\_Rescale$ is False  */ 
\end{flushleft}

This functions sets up a vector plot, and is otherwise similar to
$XGSet2D$().  The additional parameter $Z\_Label$ is described above.
Note that $PlotType$ must be set to ``vecvec''.

A call to $XGSetVec$() must be followed by one and only one call to either
$XGVector$() or $XGVectorVector$().

\begin{flushleft}
$XGVector$($x\_array$, $y\_array$, $x\_component\_data$,
$y\_component\_data$, $mpoints$, $npoints$, $color$) \\
SCALAR   *$x\_array$;    /* x position array to be plotted                         */ \\
SCALAR   *$y\_array$;    /* y positionarray to be plotted
*/ \\
SCALAR	**$x\_component\_data$;	/* x component of vectors	*/ \\
SCALAR	**$y\_component\_data$;	/* y component of vectors	*/ \\
int     *$mpoints$;    /* number of points in the x direction    */ \\
int     *$npoints$;    /* number of points in the y direction    */ \\
int      $color$;      /* plot color chosen from 0-9                   */
\end{flushleft}

\begin{flushleft}
$XGVectorVector$($x\_array$, $y\_array$, $x\_component\_data$,
$y\_component\_data$, $mpoints$, $npoints$, $color$, $x\_size$,
$x\_offset$, $y\_size$, $y\_offset$, $x\_component\_size$,
$x\_component\_offset$, $y\_component\_size$, $y\_component\_offset$ ) \\
SCALAR   *$x\_array$;    /* x position array to be plotted                         */ \\
SCALAR   *$y\_array$;    /* y position array to be plotted
*/ \\
SCALAR	**$x\_component\_data$;	/* x component of vectors	*/ \\
SCALAR	**$y\_component\_data$;	/* y component of vectors	*/ \\
int     *$mpoints$;    /* number of points in the x direction    */ \\
int     *$npoints$;    /* number of points in the y direction    */ \\
int      $color$;      /* plot color chosen from 0-9
*/ \\
int      $x\_size$;      /* number of SCALARs in the stucture which
contains the x position data	*/ \\
int      $x\_offset$;      /* number of SCALARs preceding the x field
in the structure containing the x position data		*/ \\
int      $y\_size$;      /* number of SCALARs in the structure which
contains the y position data*/ \\
int      $y\_offset$;      /* number of SCALARs preceding the y field 
in the structure containing the y position data*/ \\
int      $x\_component\_size$;      /* number of SCALARs in the stucture which
contains the x component data	*/ \\
int      $x\_component\_offset$;      /* number of SCALARs preceding the x
component field in the structure containing the x component data */ \\
int      $y\_component\_size$;      /* number of SCALARs in the structure which
contains the y component data */ \\
int      $y\_component\_offset$;      /* number of SCALARs preceding the y
component field in the structure containing the y component data */
\end{flushleft}

As in the case of $XGCurve$ (), $XGCurveVector$ () and $XGScat2D$ ()
these functions acquire data via shared memory.  Once the pointers
have been sent, the memory arrays must not be moved or freed until
program termination.  Note, however, that only one plot per frame is
allowed for three-axis plots.

\def\xlen{4in}
\begin{minipage}{\xlen}
\begin{flushright}
\rule{\xlen}{.5pt}
\end{flushright}
\end{minipage}

\begin{flushleft}
$XGSet3D$($PlotType$, $X\_Label$, $Y\_Label$, $Z\_Label$, $Theta$, $Phi$, $State$, $ulx$, $uly$, \\
\hspace{.8in} $X\_Scale$, $Y\_Scale$, $Z\_Scale$, $X\_Auto\_Rescale$, $Y\_Auto\_Rescale$, \\
\hspace{.8in} $Z\_Auto\_Rescale$, $X\_Min$, $X\_Max$, $Y\_Min$, $Y\_Max$, $Z\_Min$, $Z\_Max$) \\
char  *$PlotType$;          /* ``linlinlin'', ``linlinlog'', ``linloglog'', etc. */ \\
char  *$X\_Label$;           /* x label for the frame                            */ \\
char  *$Y\_Label$;           /* y label for the frame                            */ \\
char  *$Z\_Label$;           /* z label for the frame                            */ \\
SCALAR  $Theta$, $Phi$;       /* initial viewing angles                           */ \\
char  *$State$;             /* ``open'' or ``closed''                            */ \\
int    $ulx$, $uly$;        /* requested position of frame's upper left corner    */ \\
SCALAR  $X\_Scale$;           /* scaling factor for the x array                   */ \\
SCALAR  $Y\_Scale$;           /* scaling factor for the y array                   */ \\
SCALAR  $Z\_Scale$;           /* scaling factor for the z array                   */ \\
int    $X\_Auto\_Rescale$;    /* if True $X\_Min$ and $X\_Max$ are ingonred      */ \\
int    $Y\_Auto\_Rescale$;    /* if True $Y\_Min$ and $Y\_Max$ are ignored       */ \\
int    $Z\_Auto\_Rescale$;    /* if True $Z\_Min$ and $Z\_Max$ are ignored       */ \\
SCALAR  $X\_Min$, $X\_Max$;    /* x bounds for the plot if $X\_Auto\_Rescale$ is False  */ \\
SCALAR  $Y\_Min$, $Y\_Max$;    /* y bounds for the plot if $Y\_Auto\_Rescale$ is False  */ \\
SCALAR  $Z\_Min$, $Z\_Max$;    /* z bounds for the plot if $Z\_Auto\_Rescale$ is False  */
\end{flushleft}

This functions sets up a three-axis plot frame, and is otherwise similar to
$XGSet2D$().  Additional parameters described above for the third
dimension include $Z\_Label$, $Z\_Scale$, $Z\_Auto\_Rescale$, $Z\_Min$ and 
$Z\_Max$.

The other additional parameters specify the viewing direction for the plot
frame.  $Theta$ specifies the polar angle (0-180 degrees).  $Phi$ specifies
the azimuthal angle (0-360 degrees).  There is no perspective available in
this type of plot.

A call to $XGSet3D$() must be followed by one and only one call to
either $XGSurf$ (), $XGSurfVector$ () or $XGScat3D$ ().  

\begin{flushleft}
$XGScat3D$($x\_array$, $y\_array$, $z\_array$, $npoints$, $color$) \\
SCALAR   *$x\_array$;    /* x array to be plotted                         */ \\
SCALAR   *$y\_array$;    /* y array to be plotted                         */ \\
SCALAR   *$z\_array$;    /* z array to be plotted                         */ \\
int     *$npoints$;    /* number of points in the x, y and z direction   */ \\
int      $color$;      /* plot's color chosen from 0-9                   */
\end{flushleft}

\begin{flushleft}
$XGSurf$($x\_array$, $y\_array$, $z\_array$, $mpoints$, $npoints$, $color$) \\
SCALAR   *$x\_array$;    /* x array to be plotted                         */ \\
SCALAR   *$y\_array$;    /* y array to be plotted                         */ \\
SCALAR  **$z\_array$;    /* z array to be plotted                         */ \\
int     *$mpoints$;    /* number of points in the x direction            */ \\
int     *$npoints$;    /* number of points in the y direction            */ \\
int      $color$;      /* plot's color chosen from 0-9                   */
\end{flushleft}

\begin{flushleft}
$XGSurfVector$($x\_array$, $y\_array$, $z\_array$, $mpoints$,
$npoints$, $color$, $x\_size$, $x\_offset$, $y\_size$, $y\_offset$, $z\_size$, $z\_offset$) \\
SCALAR   *$x\_array$;    /* x array to be plotted                         */ \\
SCALAR   *$y\_array$;    /* y array to be plotted                         */ \\
SCALAR  **$z\_array$;    /* z array to be plotted                         */ \\
int     *$mpoints$;    /* number of points in the x direction            */ \\
int     *$npoints$;    /* number of points in the y direction            */ \\
int      $color$;      /* plot's color chosen from 0-9
*/ \\
int      $x\_size$;      /* number of SCALARs in the stucture which
contains the x data	*/ \\
int      $x\_offset$;      /* number of SCALARs preceding the x field
in the structure containing the x data		*/ \\
int      $y\_size$;      /* number of SCALARs in the structure which
	contains the y data*/ \\
int      $y\_offset$;      /* number of SCALARs preceding the y field
 in the structure containing the y data*/ \\
int      $z\_size$;      /* number of SCALARs in the structure which
	contains the z data*/ \\
int      $z\_offset$;      /* number of SCALARs preceding the z field
 in the structure containing the z data*/
\end{flushleft}

$XGScat3D$() adds a three-axis scatter plot to the plot frame set up by the
preceding call to $XGSet3D$().  $XGSurf$() and $XGSurfVector$() add a
three-axis wireframe/surface plot to the plot frame set up by the preceding call to $XGSet3D$().

As in the case of $XGScat2D$(),  $XGCurve$() and $XGCurveVector$(), these functions acquire data
via shared memory.  Once the pointers have been sent, the memory arrays must
not be moved or freed until program termination.  Note, however, that only
one plot per frame is allowed for three-axis plots.

\newpage

\def\xlen{4in}
\begin{minipage}{\xlen}
\begin{flushright}
\rule{\xlen}{.5pt}
\end{flushright}
\end{minipage}

\begin{flushleft}
$XGSet2DC$($PlotType$, $X\_Label$, $Y\_Label$, $Z\_Label$, $State$, $ulx$, $uly$, \\
\hspace{.8in} $X\_Scale$, $Y\_Scale$, $Z\_Scale$, $X\_Auto\_Rescale$, $Y\_Auto\_Rescale$, \\
\hspace{.8in} $Z\_Auto\_Rescale$, $X\_Min$, $X\_Max$, $Y\_Min$, $Y\_Max$, $Z\_Min$, $Z\_Max$) \\
char  *$PlotType$;          /* ``linlinlin'', ``linlinlog'', ``linloglog'', etc. */ \\
char  *$X\_Label$;           /* x label for the frame                            */ \\
char  *$Y\_Label$;           /* y label for the frame                            */ \\
char  *$Z\_Label$;           /* z label for the frame                            */ \\
char  *$State$;             /* ``open'' or ``closed''                            */ \\
int    $ulx$, $uly$;        /* requested position of frame's upper left corner    */ \\
SCALAR  $X\_Scale$;           /* scaling factor for the x array                   */ \\
SCALAR  $Y\_Scale$;           /* scaling factor for the y array                   */ \\
SCALAR  $Z\_Scale$;           /* scaling factor for the z array                   */ \\
int    $X\_Auto\_Rescale$;    /* if True $X\_Min$ and $X\_Max$ are ignored      */ \\
int    $Y\_Auto\_Rescale$;    /* if True $Y\_Min$ and $Y\_Max$ are ignored       */ \\
int    $Z\_Auto\_Rescale$;    /* if True $Z\_Min$ and $Z\_Max$ are ignored       */ \\
SCALAR  $X\_Min$, $X\_Max$;    /* x bounds for the plot if $X\_Auto\_Rescale$ is False  */ \\
SCALAR  $Y\_Min$, $Y\_Max$;    /* y bounds for the plot if $Y\_Auto\_Rescale$ is False  */ \\
SCALAR  $Z\_Min$, $Z\_Max$;    /* z bounds for the plot if $Z\_Auto\_Rescale$ is False  */
\end{flushleft}

This function creates a two-axis contour plot.  The parameters for this 
function are similar to those of $XGSet3D$(), with the exception that 
$XGSet2DC$() does not need viewing angles.

A call to $XGSet2DC$() must be followed by one and only one call to $XGCont$()
to add the contour plot to the frame.

\newpage
\begin{flushleft}
$XGCont$($x\_array$, $y\_array$, $z\_array$, $mpoints$, $npoints$, $color$) \\
SCALAR   *$x\_array$;    /* x array to be plotted                         */ \\
SCALAR   *$y\_array$;    /* y array to be plotted                         */ \\
SCALAR  **$z\_array$;    /* z array to be plotted                         */ \\
int     *$mpoints$;    /* number of points in the x direction            */ \\
int     *$npoints$;    /* number of points in the y direction            */ \\
int      $color$;      /* plot's color chosen from 0-9                   */
\end{flushleft}

This function adds a two-axis contour plot to the contour plot frame 
specified in the preceding call to $XGSet2DC$().  Note that only one
contour plot can be associated with each contour plot frame.

$XGCont$() uses shared memory to acquire its data.  Once the pointers to the
data arrays have been passed to this function, the memory must not be moved
or freed until program termination.

\def\xlen{4in}
\begin{minipage}{\xlen}
\begin{flushright}
\rule{\xlen}{.5pt}
\end{flushright}
\end{minipage}

\begin{flushleft}
$XGStructure$($numPoints$, $fillFlag$, $lineColor$, $fillColor$, $...$) \\
int 		$numPoints$;	/* number of pairs of (x,y) coordinates */ \\
STRUCT\_FILL 	$fillFlag$;	/* FILLED or HOLLOW */ \\
int		$lineColor$;	/* color of the outline */ \\
int		$fillColor$;	/* color of the inside */ \\
SCALAR		$X1, Y1$;	/* first (x,y) pair */ \\
... \\
SCALAR		$Xn, Yn$;	/* $numPoints$'th (x,y) pair */
\end{flushleft}

This function, to be called after a call to either $XGSet2D$() or
$XGSetVec$(), will set up a structured object with $numPoints$  points.  If the
first point and the last point are equal, then the object will be
closed.  Note that this procedure can be called multiple times to
define disjoint structures.
\newpage
\begin{flushleft}
$XGStructureArray$($numPoints$, $fillFlag$, $lineColor$, $fillColor$, $points$) \\
int 		$numPoints$;	/* number of pairs of (x,y) coordinates */ \\
STRUCT\_FILL 	$fillFlag$;	/* FILLED or HOLLOW */ \\
int		$lineColor$;	/* color of the outline */ \\
int		$fillColor$;	/* color of the inside */ \\
SCALAR*		$points$;	/* array containing pairs of points, i.e., $[x_1,y_1,x_2,y_2,...,x_n,y_n]$. */\\

This function is exactly like XGStructure except that you pass in an array
of points instead of using a variable argument list.
\end{flushleft}


\def\xlen{4in}
\begin{minipage}{\xlen}
\begin{flushright}
\rule{\xlen}{.5pt}
\end{flushright}
\end{minipage}

\newpage
Additional functions provided by XGrafix are described below.

\begin{flushleft}
$XGSet2DFlag$ (), $XGSetVecFlag$ (), $XGSet3DFlag$ () and
$XGSet2DCFlag$ ()
\end{flushleft}

These functions are identical to $XGSet2D$ (), $XGSetVec$ (),
$XGSet3D$ () and $XGSet2DC$ () respectively with the exception that
the ``Flag'' functions each have one additional parameter at the end
of their parameter lists which is ``int *$openFlag$''.  This integer
parameter is set to TRUE by XGrafix whenever the corresponding
diagnostic window is opened and set to FALSE when it is closed.

\begin{flushleft}
$SetUpNewVar$($spvar$, $spname$, $type$) \\
SCALAR  *$spvar$;	/* pointer to variable to be modified  */ \\
char   *$spname$;	/* name to associate with variable     */ \\
char   *$type$;		/* ``double'', ``float'', ``integer'', or ``char'' */
\end{flushleft}

This function, to be called before $XGStart$(), sets up a variable to be 
modifiable during runtime.  The parameter $spvar$ should be a pointer to 
the variable.  $spname$ is used to describe the variable, i.e. ``phase shift'' 
and $type$ specifies the type of variable.

\begin{flushleft}
$XGRead$($ptr$, $size$, $nitems$, $stream$, $type$) \\
char *$ptr$, *$type$; \\
FILE *$stream$;       \\
int $size$, $nitems$;
\end{flushleft}

The $XGRead$() function provides a platform-independent version of the 
standard C library fread function.  The parameters are identical to the those
of $fread$(); refer to your C library reference for details.  This function
makes any binary data file transportable to machines regardless of the native
byte ordering on the machine.  The parameter $type$ is only used on
UNICOS machines and takes the values ``char'', ``int'', or ``float''

\newpage

\begin{flushleft}
$XGWrite$($ptr$, $size$, $nitems$, $stream$, $type$) \\
char *$ptr$, *$type$;  \\
FILE *$stream$;        \\
int $size$, $nitems$;
\end{flushleft}

The $XGWrite$() function provides a platform-independent version of the 
standard C library fwrite function.  The parameters are identical to the those
of $fwrite$(); refer to your C library reference for details.  This function
makes any binary data file transportable to machines regardless of the native
byte ordering on the machine.The parameter $type$ is only used on
UNICOS machines and takes the values ``char'', ``int'', or ``float''

\begin{flushleft}
void $Dump$($filename$) \\
char *$filename$;
\end{flushleft}

$Dump$() provides a callback for the application to specify how to save
the state of the simulation.  This function is supplied by the application, 
and is called when the OK button is pushed within the dump dialog box.  The
parameter $filename$ contains the name of the file input in the dump dialog box.
The application can provide code to write the desired data to the file.  
Note that any operation can be performed in $Dump$() that returns control
to XGrafix.  For binary file operations, use of $XGRead$() and $XGWrite$() are 
recommended for platform independent data files.

\begin{flushleft}
void $Quit$()
\end{flushleft}

The $Quit$() function is called when the user selects the Quit button at 
runtime.  The function must be supplied by the application, and can perform
an operation required just before exit, such as saving data, etc.

\begin{flushleft}
int $C\_KillGraphicsProc$()
\end{flushleft}

The $C\_KillGraphicsProc$() terminates the X display components of a
simulation leaving the program to iterate the XGMainLoop without
displaying the control panel or any diagnostic windows.  The function
is executed by choosing the ``Kill X'' option from the ``Quit'' menu
or sending the program a ``USR1'' signal with the ``kill'' command.
The X display may be reactivated by sending the program another
``USR1'' signal.  In order to restore the X frontend, a file named
``.xgdisplay'' in the users \$HOME directory containing a display name
(i.e. my\_machine.universtiy.edu:0) to which the windows will be sent
must exist.  Of course this machine must give permision for the
host machine to display X there.

{\bf Warning:} This routine is
extremely volatile.  It should not be used without first saving the
state of the simulation.  Also it is not currently supported in XOOPIC.

\def\xlen{4in}
\begin{minipage}{\xlen}
\begin{flushright}
\rule{\xlen}{.5pt}
\end{flushright}
\end{minipage}

Key bindings are built into XGrafix to facilitate usage.  In all dialog boxes,
the {\bf UP ARROW} and {\bf DOWN ARROW} can be used to maneuver between 
options.  {\bf TAB} will invoke the $Apply$ button (if one exists),
{\bf RETURN} will invoke the $Ok$ button, {\bf SPACE} will select an On/Off 
button, and {\bf ESCAPE} will invoke the
$Cancel$ button.  In the $Diagnostics$ dialog box, {\bf TAB} will start
a search based on the contents of the entry, {\bf ENTER} will invoke the
$Ok$ button, and the {\bf UP ARROW}, {\bf DOWN ARROW}, {\bf PAGE UP}, and
{\bf PAGE DOWN} keys can be used to scroll the listbox.

\end{document}














